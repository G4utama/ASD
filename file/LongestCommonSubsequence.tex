\section{Longest Common Subsequence}
Longest Common Subsequence (LCS) e' un problema di ottimizzazione, ovvero si cerca di trovare una soluzione ottima.
\subsection{Problema di Ottimizzazione}
\begin{mdframed}
    Date 2 stringhe $X,Y$ determina $Z$ tale che:
    \begin{itemize}
        \item $Z$ è sottosequenza di $X$ e $Y$
        \item $Z$ è la più lunga tra tutte le sottosequenze comuni
    \end{itemize}
\end{mdframed}
Avendo complessità esponenziale, si cerca di individuare una struttura ricorsiva, cioè una proprietà di sottostruttura. La LCS deve "nascondere" al suo interno LCS di qualche stringa più piccola di $X$ e $Y$.
\begin{equation*}
\begin{rcases}
    X = \langle X',a \rangle \\
    Y = \langle Y',b \rangle
\end{rcases} \Rightarrow Z \text{ è la stringa più lunga tra } \verb|LCS|(X',Y) \text{ e } \verb|LCS|(X,Y')
\end{equation*}
Spazio sottoproblemi: $S = \{ \verb|LCS|(X_i,Y_j): 0 \leq i \leq m, 0 \leq j \leq n \} \Rightarrow |S| = (m+1)(n+1)$

\subsection{Proprietà di Sottostruttura}
Ottima per il sottoproblema \verb|LCS|$(X_i,Y_j)$
Dati:
\begin{itemize}
    \item $X_i = \langle x_1,x_2,\dots,x_i \rangle$
    \item $Y_i = \langle y_1,y_2,\dots,y_j \rangle$
\end{itemize}
Sia $Z = \langle z_1,z_2,\dots,z_k \rangle = \verb|LCS|(X_i,Y_j)$
\begin{itemize}
    \item caso base: $i = j = 0 \Rightarrow Z = \varepsilon$
    \item $i,k>0$: se $x_i = y_i$ \\
    $\Rightarrow z_k = x_i(= y_i)$ \\
    $\Rightarrow Z_{k-1} = \verb|LCS|(X_{i-1},Y_{j-1})$
    \item $i,j>0$: se $x_i \not= y_j$ \\
    $\Rightarrow Z$ è la stringa di lunghezza max tra \verb|LCS|$(X_{i-1},Y_j)$ e \verb|LCS|$(X_i,Y_{j-1})$
\end{itemize}

\subsection{Ricorrenza sui Costi}
La scrittura della ricorrenza sui costi è il secondo passo per costruire un algoritmo di programmazione dinamica.
\begin{equation*}
l(i,j) = |\verb|LCS|(X_i,Y_j)| = 
\begin{cases}
    \text{caso base: se } i=0, j=0 \qquad\;\; 0 \\
    \text{caso 1: se } i,j>0, x_i = x_j \qquad\; l(i-1,j-1)+1 \\
    \text{caso 2: se } i,j>0, x_i \not= x_j \qquad \max(l(i,j-1),l(i-1,j))
\end{cases}
\end{equation*}
Ci interessa calcolare $l(m,n)$

\subsection{Modello di Costo: confronto tra caratteri}
\begin{equation*}
    T(n,m) =
    \begin{cases}
        \text{se } n,m=0 \qquad 0 \\
        \text{se } n,m>0 \qquad T(n-1,m) + T(n,m-1) + 1 \\
    \end{cases}
\end{equation*}

\newpage
\subsection{Pseudocodice}
\begin{mdframed}
\begin{lstlisting}[mathescape=true]
LCS(X,Y)
1   m = X.length
2   n = Y.length
3   for i=0 to m
4       L[i,0] = 0
5   for j=0 to n
6       L[0,j] = 0
7   for i=1 to m
8       for j=1 to n
9           if $x_i$ = $y_j$
10              L[i,j] = L[i-1,j-1] + 1
11              B[i,j] = '$\nwarrow$'
12           else if L[i-1,j] >= L[i,j-1]
13              L[i,j] = L[i-1,j]
14              B[i,j] = '$\uparrow$'
15           else
16              L[i,j] = L[i,j-1]
17              B[i,j] = '$\leftarrow$'
18  return (L[m,n],B)
\end{lstlisting}
\end{mdframed}
\paragraph{Complessità} $T(m,n) = \Theta (m \cdot n)$	
\begin{mdframed}
\begin{lstlisting}[mathescape=true]
PRINT-LCS(B,X,i,j)
1   if i=0 or j=0
2       return $\varepsilon$
3   if B[i,j] = '$\nwarrow$'
4       PRINT-LCS(B,X,i-1,j-1)
5       print $x_i$
6   else if B[i,j] = '$\leftarrow$'
7       PRINT-LCS(B,X,i,j-1)
8   else
9       PRINT-LCS(B,X,i-1,j)
\end{lstlisting}
\end{mdframed}
\paragraph{Complessità} $\Theta(m) = \Theta(|\verb|LCS||)$ \\~\\
In particolare, per la LCS:
\begin{itemize}
    \item La prima riga e la prima colonna sono sempre inizializzate a $0$
    \item Appena si ha un nuovo match, si incrementa il conteggio di $1$
    \item Le righe successive controllano sempre la riga precedente verificando se si ha avuto un match in quella posizione, oppure verificando in diagonale a sx se si ha avuto un match
    \begin{itemize}
        \item se si ha avuto un match per la stessa lettera sulla stessa riga, ma in diagonale a sx si aveva 0, si ha 1 sulla riga attuale
        \item se si ha avuto un match per la stessa lettera sulla stessa riga, ma in alto o in diagonale a sx si ha lo stesso numero, allora si incrementa perché si trova un match
    \end{itemize}
    \item Se si ha la freccia in alto, non si hanno avuti match
    \item Se si ha la freccia che va a sx, si ha avuto un match sulla stessa riga in posizione precedente
\end{itemize}

\subsection{Codice Memoizzato}
\begin{mdframed}
\begin{lstlisting}[mathescape=true]
INIT-LCS(X,Y)
1   m = X.length
2   n = Y.length
3   if m = 0 or n = 0
4       return 0
5   for i=0 to m
6       L[i,0] = 0
7   for j=0 to n
8       L[0,j] = 0
9   for i=1 to m
10       for j=1 to n
11           L[i,j] = -1
12  return R-LCS(X,Y,m,n)
\end{lstlisting}
\end{mdframed}
\begin{mdframed}
\begin{lstlisting}[mathescape=true]
R-LCS(X,Y,i,j)
1   if L[i,j] = -1
2       if $x_i$ = $y_j$
3           L[i,j] = R-LCS(X,Y,i-1,j-1)
4       else if R-LCS(X,Y,i-1,j) >= R-LCS(X,Y,i,j-1)
5           L[i,j] = L[i-1,j]
6       else
7           L[i,j] = L[i,j-1]
8   return L[i,j]
\end{lstlisting}
\end{mdframed}
In particolare, si ha:
\begin{itemize}
    \item una procedura di inizializzazione, per riempire la tabella bidimensionale di analisi messo a $0$ all'inizio per le celle $i$ e $j$, per poi porre, al secondo giro di inizializzazione $-1$.
    \item una procedura che controlla che tutte le celle siano a $-1$ e parte dalla fine della tabella fino all'inizio iterativamente (vuol dire che quei valori devono essere ancora esaminati) e:
    \begin{itemize}
        \item se i valori sono uguali procedo in diagonale
        \item se il valore della riga precedente/stessa colonna è $>=$ rispetto al valore della stessa riga e colonna precedente, si sposta verticalmenteo
        \item altrimenti, si sposta lateralmente
    \end{itemize}
\end{itemize}
\begin{equation*}
    b(i,j) =
    \begin{cases}
        \nwarrow \quad \text{se } x_i = y_i \\
        \leftarrow \quad \text{se } x_i \not= x_j \text{ e } \max = LCS(i,j-1) \\
        \uparrow \quad\;\; \text{se } x_i \not= y_j \text{ e } \max = LCS(i-1,j)
    \end{cases}
\end{equation*}
\paragraph{Complessità} $\Theta(m \cdot n)$ \\~\\
In generale, se $Y$ è suffisso di $n \leq m$ caratteri di $X$, la complessità di \verb|R-LCS| è $T_{\verb|R-LCS|}(m,n) = n$

\newpage