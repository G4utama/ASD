\section{Longest Common Subsequence}
Longest Common Subsequence (LCS) e' un problema di ottimizzazione, ovvero si cerca di trovare una soluzione ottima.
\subsection{Problema di Ottimizzazione}
\begin{mdframed}
    Date 2 stringhe $X,Y$ determina $Z$ tale che:
    \begin{itemize}
        \item $Z$ è sottosequenza di $X$ e $Y$
        \item $Z$ è la più lunga tra tutte le sottosequenze comuni
    \end{itemize}
\end{mdframed}
Avendo complessità esponenziale, si cerca di individuare una struttura ricorsiva, cioè una proprietà di sottostruttura. La LCS deve "nascondere" al suo interno LCS di qualche stringa più piccola di $X$ e $Y$.
\begin{equation*}
\begin{rcases}
    X = \langle X',a \rangle \\
    Y = \langle Y',b \rangle
\end{rcases} \Rightarrow Z \text{ è la stringa più lunga tra } \verb|LCS|(X',Y) \text{ e } \verb|LCS|(X,Y')
\end{equation*}
Spazio sottoproblemi: $S = \{ \verb|LCS|(X_i,Y_j): 0 \leq i \leq m, 0 \leq j \leq n \} \Rightarrow |S| = (m+1)(n+1)$

\subsection{Proprietà di Sottostruttura}
Ottima per il sottoproblema \verb|LCS|$(X_i,Y_j)$
Dati:
\begin{itemize}
    \item $X_i = \langle x_1,x_2,\dots,x_i \rangle$
    \item $Y_i = \langle y_1,y_2,\dots,y_j \rangle$
\end{itemize}
Sia $Z = \langle z_1,z_2,\dots,z_k \rangle = \verb|LCS|(X_i,Y_j)$
\begin{itemize}
    \item caso base: $i = j = 0 \Rightarrow Z = \varepsilon$
    \item $i,k>0$: se $x_i = y_i$ \\
    $\Rightarrow z_k = x_i(= y_i)$ \\
    $\Rightarrow Z_{k-1} = \verb|LCS|(X_{i-1},Y_{j-1})$
    \item $i,j>0$: se $x_i \not= y_j$ \\
    $\Rightarrow Z$ è la stringa di lunghezza max tra \verb|LCS|$(X_{i-1},Y_j)$ e \verb|LCS|$(X_i,Y_{j-1})$
\end{itemize}

\subsection{Ricorrenza sui Costi}
La scrittura della ricorrenza sui costi è il secondo passo per costruire un algoritmo di programmazione dinamica.
\begin{equation*}
l(i,j) = |\verb|LCS|(X_i,Y_j)| = 
\begin{cases}
    \text{caso base: se } i=0, j=0 \qquad\;\; 0 \\
    \text{caso 1: se } i,j>0, x_i = x_j \qquad\; l(i-1,j-1)+1 \\
    \text{caso 2: se } i,j>0, x_i \not= x_j \qquad \max(l(i,j-1),l(i-1,j))
\end{cases}
\end{equation*}
Ci interessa calcolare $l(m,n)$

\subsection{Modello di Costo: confronto tra caratteri}
\begin{equation*}
    T(n,m) =
    \begin{cases}
        \text{se } n,m=0 \qquad 0 \\
        \text{se } n,m>0 \qquad T(n-1,m) + T(n,m-1) + 1 \\
    \end{cases}
\end{equation*}
