\section{Shortest Palindrome Completion}
\begin{mdframed}
    \textbf{Def} \quad Una stringa $Z = \langle z_1,z_2,\ldots,z_n \rangle$ è palindroma se $z_{1+h} = z_{m-h} \quad (\forall\; 0 \leq h \leq m-1)$
\end{mdframed}
\paragraph{Problema:} Data $X = \langle x_1,x_2,\ldots,x_n \rangle$ un complemento palindromo di $X$ è una stringa $Z = CP(X) = \langle z_1,z_2,\ldots,z_m \rangle$ con $m \geq n$ tale che:
\begin{itemize}
    \item $Z$ e' palindroma
    \item $X$ è sottosequenza di $Z$
\end{itemize}
\subsection{Proprietà di Sottostruttura Ottima}
Casi:
\begin{itemize}
    \item $i=j$ \\~\\
          $X = \langle x_i \rangle$ \\
          $CP(X_{i \ldots j}) = X_{i \ldots j}$ \\
          $|CP(X_{i \ldots j})| = |X_{i \ldots j}|$
    \item $j = i+1$ \\~\\
          $X_{i \ldots j} = \langle x_i,x_{i+1} \rangle$
    \item $X_{i \ldots j} = \langle x_i,x_{i+1},\ldots,x_j \rangle$
    \begin{itemize}
        \item $x_i = x_j$ \\~\\
              $Z = CP(X_{i \ldots j})$ \\
              $z_1 = z_k = x_i (= x_j)$ \\
              $Z_{2 \ldots k-1} = CP(X_{i+1 \ldots j-1})$
        \item $x_i \not= x_j$
        \begin{itemize}
            \item $z_1 = z_k = x_i$ \\
                  $Z_{2 \ldots k-1} = CP(X_{i+1 \ldots j})$
            \item $z_1 = z_k = x_j$ \\
                  $Z_{2 \ldots k-1} = CP(X_{i \ldots j-1})$
        \end{itemize}
    \end{itemize}
\end{itemize}

\subsection{Ricorrenza sulle Lunghezze}
Definisco:
\begin{itemize}
    \item $l(i,j) = |CP(X_{i \ldots j})|$
    \item $l(i,j) = \begin{cases}
        \text{se } i = j \qquad\qquad\qquad\;\;\;\;\; 1 \\
        \text{se } j = i+1, x_i = x_j \qquad 2\\
        \text{se } j = i+1, x_i \not= x_j \qquad 3\\
        \text{se } j > i+1, x_i = x_j \qquad 2 + l(i+1,j-1)\\
        \text{se } j > i+1, x_i \not= x_j \qquad 2 + \min\{l(i+1,j),l(i,j-1)\}
    \end{cases}$
\end{itemize}

\newpage
\subsection{Pseudocodice}
\begin{mdframed}
\begin{lstlisting}[language=C]
SPC(X)
1   n = X.length
2   for i=1 to n-1
3       L[i,j] = 1
4       if x_i = x_{i+1}
5           L[i,i+1] = 2
6       else
7           L[i,i+1] = 3
8   L[n,n] = 1
9   for l=3 to n
10      for i=1 to n-l+1
11          j = i+l-1
12          if x_i = x_j
13              L[i,j] = 2 + L[i+1,j-1]
14          else
15              L[i,j] = 2 + min(L[i+1,j],L[i,j-1])
16  return L[1,n]
\end{lstlisting}
\end{mdframed}
Per l’algoritmo:
\begin{itemize}
    \item salvo la lunghezza iniziale della parola
    \item inizializzo l'array delle lunghezze e copro i casi base (stringhe con 2 caratteri uguali a stringhe consecutive (metto 2) o stringhe con 2 caratteri uguali non consecutive (metto 3))
    \item procedo con una scansione diagonale (con $l$ che parte da 3 avendo esaminato i casi base palindromo e assicurandomi di non avere stringa palindroma di 1-2 caratteri)
    \item l'altro ciclo esclude tutte le stringhe già controllate (lunghezza 1/2)
    \item se le stringhe sono uguali, continuo a cercare in diagonale
    \item altrimenti, scelgo la stringa di lunghezza minima (perché cerco di ottimizzare il problema e quindi scelgo la stringa più corta da controllare) a sinistra e sotto
\end{itemize}
\paragraph{Complessità} $\Theta(n^2)$
\begin{equation*}
    \sum_{i=1}^{n-1} + \sum_{l=3}^{n} \sum_{i=1}^{n-l+1} =
    \sum_{i=1}^{n-1} + \sum_{l=3}^{n}(n-l+1) =
    \sum_{i=1}^{n-1} + \sum_{i=2}^{n-1}(n-i) =
    \sum_{i=1}^{n-1}(n-i) =
    \sum_{j=1}^{n-1}j =
    \frac{n(n-1)}{2} =
    \Theta(n^2)
\end{equation*}

\newpage