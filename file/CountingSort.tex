\section{Counting Sort}
Il Counting Sort è una tecnica di ordinamento stabile, utilizzata per ordinare gli oggetti in base alle chiavi che sono piccoli numeri. Conta il numero di chiavi i cui valori sono uguali. Questa tecnica di ordinamento è efficiente quando la differenza tra le diverse chiavi non è così grande, altrimenti può aumentare la complessità dello spazio.

\subsection{Funzionamento}
\begin{enumerate}
    \item Scopri l'elemento $max$ dall'array specificato. \\
    \begin{tabular}{c}
        $max$ \\ 
    \end{tabular}
    \\
    \begin{tabular}{|c|}
        \hline
        \; 8 \; \\ 
        \hline
    \end{tabular}
    \begin{tabular}{|c|c|c|c|c|c|c|}
        \hline
        4 & 2 & 2 & 8 & 3 & 3 & 1 \\ 
        \hline
    \end{tabular}

    \item Inizializzare una matrice di lunghezza $max+1$ con tutti gli elementi $0$. Questa matrice viene utilizzata per memorizzare il conteggio degli elementi nella matrice.
    \item[]
    \begin{tabular}{|c|c|c|c|c|c|c|c|c|}
        \hline
        0 & 0 & 0 & 0 & 0 & 0 & 0 & 0 & 0 \\ 
        \hline
    \end{tabular}
    \\
    \begin{tabular}{ccccccccc}
        0 & 1 & 2 & 3 & 4 & 5 & 6 & 7 & 8 \\ 
    \end{tabular}

    \item Memorizza il conteggio di ciascun elemento nel rispettivo indice nell'array count.
    \item[]
    \begin{tabular}{|c|c|c|c|c|c|c|c|c|}
        \hline
        0 & 1 & 2 & 2 & 1 & 0 & 0 & 0 & 1 \\ 
        \hline
    \end{tabular}
    \\
    \begin{tabular}{ccccccccc}
        0 & 1 & 2 & 3 & 4 & 5 & 6 & 7 & 8 \\ 
    \end{tabular}

    \item Memorizzare la somma cumulativa degli elementi dell'array count. Aiuta a posizionare gli elementi nell'indice corretto dell'array ordinato.
    \item[]
    \begin{tabular}{|c|c|c|c|c|c|c|c|c|}
        \hline
        0 & 1 & 3 & 5 & 6 & 6 & 6 & 6 & 7 \\ 
        \hline
    \end{tabular}
    \\
    \begin{tabular}{ccccccccc}
        0 & 1 & 2 & 3 & 4 & 5 & 6 & 7 & 8 \\ 
    \end{tabular}

    \item Individuare l'indice di ogni elemento della matrice originale nell'array count. Questo dà il conteggio cumulativo.
    \item[]
    \begin{tabular}{|c|c|c|c|c|c|c|}
        \hline
        1 & 2 & 2 & 3 & 3 & 4 & 8 \\ 
        \hline
    \end{tabular}
    \\
    \begin{tabular}{ccccccc}
        0 & 1 & 2 & 3 & 4 & 5 & 6 \\ 
    \end{tabular}
    
\end{enumerate}

\subsection{Pseudocodice}
\begin{mdframed}
\begin{lstlisting}[language=C]
COUNTING-SORT(A,B,k)
1   C[0..k] = 0
2   for j = 1 to A.length
3       C[A[j]] = C[A[j]] + 1
4   for i = 1 to k
5       C[i] = C[i-1] + C[i]
6   for j = A.length to 1
7       B[C[A[j]]] = A[j]
8       C[A[j]] = C[A[j]] - 1
\end{lstlisting}
\end{mdframed}

\subsection{Complessità}
\begin{lstlisting}[mathescape=true]
C[0..k] = 0                 $\Theta(k)$
for j = 1 to A.length       $\Theta(n)$
    ...
for i = 1 to k              $\Theta(k)$
    ...
for j = A.length to 1       $\Theta(n)$
    ...
\end{lstlisting}
Somma $\Theta(n+k)$ con $k=\Theta(1) \Rightarrow \Theta(n)$

\newpage