\section{Radix Sort}
Il Radix Sort è un algoritmo di ordinamento che ordina gli elementi raggruppando prima le singole cifre tutte nella stessa attuale posizione. Quindi, ordina gli elementi in base al loro ordine crescente/decrescente. L'idea è quella di ordinare cifra per cifra con un algoritmo stabile e risolvere i problemi di memoria di Counting Sort.

\subsection{Pseudocodice}
\begin{mdframed}
\begin{lstlisting}[language=C]
Radix-SORT(A,d)
1   for j = 1 to d
2       COUNTING-SORT(A,j)
\end{lstlisting}
\end{mdframed}

\subsection{Correttezza}
\paragraph{Invariante:} $A^{j-1}$ ordinato
\paragraph{Inizializzazione:} $j=1, A^{j-1} = A^0$
\paragraph{Mantenimento:} $A^{j-1}$ ordinato e rodino rispetto a $j$

\subsection{Complessità}
\begin{itemize}
    \item $d$ volte Counting-Sort $\Theta(n+b) \Rightarrow \Theta(d(n+b)) = \Theta(n)$
    \item $d$ cifre $\Theta(1)$
    \item $b$ base $\Theta(n)$
\end{itemize}
$\begin{rcases}
    m \text{ bit} \Rightarrow m = O(\log n) \\
    r \text{ bit per cifra} \Rightarrow r = O(\log_2 n) \\
    \text{base } 2^r
\end{rcases}
\Rightarrow \Theta(\frac{m}{r}(m+2^r)) = \Theta(\frac{m}{\log n}(m+2^{\log n})) = \Theta(\frac{m}{\log n}n) = \Theta(n)$

\newpage