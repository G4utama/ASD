\section{RedBlack Tree}
I RB-Tree sono ABR i cui nodi hanno un bit extra riservato ad un campo colore $x.color$, che può essere: $red$ per il rosso, $black$ per il nero. \\~\\
Un RB-Tree soddisfa le seguenti proprietà:
\begin{itemize}
    \item Proprietà RB: ogni nodo è colorato, rosso o nero
    \item Proprietà della radice: la radice è nera
    \item Proprietà delle foglie: ogni foglia (NULL) è nera
    \item Proprietà del rosso: se un nodo rosso ha dei figli, questi sono sempre neri
    \item Proprietà Depth: per ogni nodo, qualsiasi percorso semplice da questo nodo a qualsiasi foglia discendente ha la stessa profondità nera (il numero di nodi neri)
\end{itemize}
\begin{center}
    \begin{tabular}{c}
        \\ \includegraphics[width=0.7\textwidth]{image/RB-Tree.png} \\ \\
    \end{tabular}
\end{center}
Le operazioni degli RB-Tree saranno chiaramente simili a prima, in particolare:
\begin{itemize}
    \item \verb|Search|
    \item \verb|Max|
    \item \verb|Min|
    \item \verb|Successor|
    \item \verb|Precessor| (se $x$ ha 2 figli, il suo predecessore è il valore max nel suo sottoalbero sx e il suo successore il valore min nel suo sottoalbero dx. Se non ha un figlio a sx, il predecessore di un nodo è il suo primo antenato a sx)
    \item \verb|Insert| (difficile mantenere la colorazione)
    \item \verb|Delete| (difficile mantenere la colorazione)
\end{itemize}
La complessità delle operazioni è sempre data dall'altezza $h$ dell'albero, come $h = O(\log(n))$ con $h \leq 2 \log_2 (n+1)$. \\~\\
