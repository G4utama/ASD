\section{Quick Sort}
L'algoritmo di ordinamento più utilizzato e generalmente più efficiente con $O(n^2)$ come caso pessimo ed $O(n\log n)$ come caso medio e migliore è proprio quicksort.
Viene definito algoritmo "in-place", cioè un algoritmo che non ha bisogno di spazio extra e produce un output nella stessa memoria che contiene i dati.

Anche questo algoritmo è basato sul paradigma divide-et-impera, infatti, per ordinare $A[p, r]$:
\begin{itemize}
    \item Divide:
    \begin{itemize}
        \item sceglie un pivot $x$ in $A[p, r]$
        \item partiziona in $A[p, q-1] \leq x$ e $A[q+1, r] \geq x$
    \end{itemize}
    \item Impera:
    \begin{itemize}
        \item ricorre su $A[p, q-1]$ e su $A[q+1, r]$
    \end{itemize}
\end{itemize}

\subsection{Pseudocodice}
\begin{mdframed}
\begin{lstlisting}[language=C]
QUICKSORT(A,p,r)
1   q = PARTITION(A,p,r)
2   QUICKSORT(A,p,q-1)
3   QUICKSORT(A,q+1,r)
\end{lstlisting}
\end{mdframed}

\begin{mdframed}
\begin{lstlisting}[language=C]
PARTITION(A,p,r)
1   x = A[r]
2   i = p - 1
3   for j = p to r - 1
4       if A[j] <= x
5           i = i + 1
6           A[i] <-> A[j]
7   A[i + 1] <-> A[r]
8   return i + 1
\end{lstlisting}
\end{mdframed}

\subsection{Passaggi}
Per ordinare un array, seguite i passaggi seguenti:
\begin{enumerate}
    \item Si sceglie come pivot un qualsiasi valore di indice dell'array.
    \item Quindi si partiziona l'array in base al pivot.
    \item Quindi si esegue un Quicksort ricorsivo della partizione di sinistra.
    \item Successivamente, si esegue un Quicksort ricorsivo della partizione corretta.
\end{enumerate}

Diamo un'occhiata più da vicino alla partizione di questo algoritmo:
\begin{enumerate}
    \item Si sceglie un pivot qualsiasi, ad esempio il valore più alto dell'indice.
    \item Si prenderanno due variabili per puntare a sinistra e a destra dell'elenco, escludendo il pivot.
    \item La sinistra punterà all'indice più basso e la destra all'indice più alto.
    \item Ora si spostano a destra tutti gli elementi maggiori del pivot.
    \item Quindi si sposteranno tutti gli elementi più piccoli del perno nella partizione di sinistra.
\end{enumerate}

\newpage