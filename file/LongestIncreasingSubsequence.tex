\section{Longest Increasing Subsequence}
\begin{mdframed}
    \textbf{Def} \quad Dato un alfabeto $\sum$ totalmente ordinato ($\forall\; a,b \in \sum \quad a<b, a=b, a>b$) e dato $X = \langle x_1,x_2,\ldots,x_n \rangle$, si dice che $Z = \langle z_1,z_2,\ldots,z_k \rangle$ è sottosequenza crescente di $X (Z = IS(X))$
\end{mdframed}

\subsection{Problema di Ottimizzazione}
Determinare la più lunga sottosequenza crescente di $X(Z = LIS(X))$
\begin{mdframed}
    \textbf{Def} \quad $Z = \overline{LIS}(X_i)$ è la più lunga tra le $IS(X_i)$ con $Z = \langle z_1,z_2,\ldots,z_k \rangle = \langle x_{i_1},x_{i_2},\ldots,x_{i_k} \rangle$ con $i_k = i$
\end{mdframed}
Casi:
\begin{itemize}
    \item \textbf{fortunato:} $Z_k < X_{i+1} \Rightarrow LIS(X_{i_1}) = \langle Z^i, X_{i+1} \rangle$
    \item \textbf{sfortunato:} $Z_k \geq X_{i+1} \Rightarrow LIS(X_{i_1}) = \langle Z^i \rangle \Rightarrow$ NO perché $X_{i+1}$ è una $LIS(X^i) \Rightarrow |LIS(X_{i+1})| = |LIS(X_i)| + 1$
\end{itemize}

\subsection{Proprietà di Sottostruttura Ottima}
\begin{itemize}
    \item caso base: $\overline{LIS}(X_1) = \langle x_1 \rangle (= LIS(X_1))$
    \item caso $i>1$
    \begin{itemize}
        \item $\forall\; j : 1 \leq j \leq i, \quad x_j \geq x_i$ \\
              $\overline{LIS}(X_i) = \langle x_i \rangle (= LIS(X_i))$
        \item $\exists\; \overline{j} : 1 \leq n \overline{j} \leq i, \quad x_{\overline{j}} < x_i$ \\
              $|\overline{LIS}(X_i)| \geq 2$ \\
              $\overline{LIS}(X_i) = \langle z_1,z_2, \ldots, z_k \rangle = \langle Z_{k-1}, x_i \rangle$ con $Z_{k-1} : |Z_{k-1}| = \max_{1 \leq j \leq i} \{\overline{LIS}(X_j) : x_j < x_i\}$
    \end{itemize}
\end{itemize}

\subsection{Ricorrenza sui Costi}
Definisco;
\begin{itemize}
    \item $l(i) = |\overline{LIS}(X_i)|$
    \item $l(i) = \begin{cases}
        \text{se } i=1 \qquad 1 \\
        \text{se } i>1 \qquad 1 + \max_{1 \leq j < i}\{l(j): x_j < x_i\}
    \end{cases}$
\end{itemize}
Per costruire la sottosequenza:
\begin{equation*}
    \overline{LIS}(X_i) = 
    \begin{cases}
        \langle x_i \rangle \\
        \langle \overline{LIS}(X_{\overline{j}}),x_i \rangle \qquad (1 \leq \overline{j} < i)
    \end{cases}
\end{equation*}
Informazioni addizionali:
\begin{itemize}
    \item $prev(i) = \begin{cases}
        0 \\
        \overline{j}
    \end{cases}$
    \item $len = \max _{1 \leq i \leq n} \{l(i)\}$
    \item $end = i \qquad \overline{LIS}(X_i) = LIS(X)$
\end{itemize}

\newpage
La LIS è quindi composta da $\langle \dots, X_(prev(prev(i))), X_(prev), X_i \rangle$ \\
Non possiamo usare un algoritmo ricorsivo, in quanto abbiamo diversi sottoproblemi ripetuti.
Scriviamo quindi l'algoritmo bottom-up, a partire da $\overline{LIS}(X_1)$.
\begin{mdframed}
\begin{lstlisting}[language=C]
LIS(x)
1   L[1] = 1
2   len = 1
3   end = 1
4   prev[1] = 0
5   for i = 2 to n
6       L[i] = 1
7       prev[i] = 0
8       for j =1 to i-1
9           if x_j < x_i
10              if L[i] < 1 + L[j]
11                  L[i] = 1 + L[j]
12                  prev[i] = j
13      if len < L[i]
14          len = L[i]
15          end = i
16  return (len, prev, end)
\end{lstlisting}
\end{mdframed}
\begin{itemize}
    \item inizializzo tutti i valori di ricerca della sottosequenza
    \item nel primo ciclo, inizializzo la lunghezza dell'elemento attuale e inizializzo l'elemento precedente
    \item nel secondo ciclo, che parte dalla fine per poter ottimizzare il proprio calcolo e avere già la soluzione ottimale partendo dal primo elemento, confronto l'elemento i-esimo con l'elemento j-esimo
    \item se fosse minore, si salva l'elemento attuale j-esimo nell'array i-esimo e si salva la lunghezza raggiunta e dove si ferma
\end{itemize}
\begin{mdframed}
\begin{lstlisting}[language=C]
R-PRINT(X,prev,i)
1   if prev[i] != 0
2       R-PRINT(X,prev,prev[i])
3   print(x_i)
\end{lstlisting}
\end{mdframed}
\paragraph{Complessità} $\sum_{i=2}^n \sum_{j=1}^{i\;1}1 = \sum_{i=2}^n (i-1) = \frac{n(n-1)}{2} = \Theta(n^2)$ \\~\\
La pratica è questa:
\begin{itemize}
    \item considero il mio array delle sottostringhe e cerco di trovare le sottosequenze che crescono
    \item nell’array delle lunghezze, consideriamo che noi teniamo traccia delle sottosequenze attuali che si incrementano, considerano che teniamo solo le sottosequenze che, a parità di numero di elementi, hanno un valore nella stessa posizione minore. Non avremo mai due sottosequenze che si incrementano con lo stesso numero di elementi
    \item noi confrontiamo gli elementi della sottostringa uno per uno; l'array delle lunghezze si aggiorna "per riga", quindi, salviamo solo le posizioni in cui la lunghezza della sottostringa sottostante aumenta, tale da poter individuare facilmente qual è la sottosequenza migliore
\end{itemize}

\newpage